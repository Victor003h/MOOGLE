\documentclass{beamer}
%\usepackage[utf8]{inputenc}
\usepackage[spanish]{babel}
\usepackage{amsmath}
\usepackage{amsfonts}
\usepackage{amssymb}
\usepackage{graphicx}
\usetheme{Madrid}

\title{Moogle}
\author{Victor Hugo Pacheco Fonseca}
\date{19 de julio del 2023}
\institute[UH]{Universidad de La Habana\\Matcom}

\begin{document}
	\begin{frame}[plain]
		\maketitle
	\end{frame}
	
	\section{Introduccion}
	\begin{frame}{Introduccion}
		Moogle! es una aplicación totalmente original cuyo propósito es buscar inteligentemente un texto en un conjunto de documentos.
		
		Es una aplicación web, desarrollada con tecnología .NET Core 6.0, específicamente usando Blazor como framework web para la interfaz gráfica, y en el lenguaje csharp.
	\end{frame}
	\section{Indicaciones para la ejecucion del proyecto}
	\begin{frame}{Indicaciones para la ejecucion del proyecto}
		Para ejecutar el programa de forma adecuada se debe seguir los siguentes
		pasos:
			\begin{enumerate}
			\item Debe copiar los documentos en lo que se quiere buscar en la carpeta 'Content', en formato .txt
			\item Luego desde carpeta raíz ejecutar el comando según el sistema operativo con que se está
			trabajando
		\end{enumerate}
			Para brindar mayor exactitud a la consulta, el proyecto cuenta con un corrector de palabras, por
		si el usuario escribió incorrectamente alguna palabra.
	\end{frame}
\end{document}